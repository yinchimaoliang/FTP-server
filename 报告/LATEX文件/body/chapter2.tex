\section{协议的具体实现}
\subsection{客户端代码实现}

\subsubsection{总体框架}
我们在窗口类初始化的时候调用另一个封装好的类myclient,用来绑定端口以及处理交互。在窗口类中主要负责实现点击按钮的具体功能以及调用线程持续监听消息。而在myclient 中负责处理接收消息后的处理,以及根据接收到的消息发送对应的信息。
\subsubsection{多线程机制}
我们在窗口类初始化的时候即调用另一个线程,在另一个线程中调用handleReceive函数,使得myclient能够持续接收到消息。

\begin{figure}[H]
  \centering
  % Requires \usepackage{graphicx}
  \includegraphics[width=0.8\linewidth]{figure/multiposix}\\
  \caption{多线程}
\end{figure}

相应代码如下:

\begin{flushleft}
\begin{lstlisting}[language={C++},basicstyle=\footnotesize]
//线程调用函数
DWORD WINAPI CClientDlg::handleReceive(LPVOID lpParameter)
{
	myclient* t = (myclient*)lpParameter;
	CString log;
	while(1)
	{
		log.Empty();
		t->handleInteraction();
		CClientDlg* dlg=(CClientDlg*)AfxGetApp()->GetMainWnd(); //获取主窗口
		dlg->m_log+=log;
		dlg->GetDlgItem(IDC_LOG)->SetWindowTextW(dlg->m_log);
	}
	return (DWORD)NULL;
}
\end{lstlisting}
\end{flushleft}

\subsubsection{按钮处理函数}
当用户按下获取文件目录按钮时,客户端向服务器发送type为APPLY\_FILE\_DIC的消息,并且在log中显示对应信息,代码如下:


\begin{flushleft}
\begin{lstlisting}[language={C++},basicstyle=\footnotesize]
//点击获取共享目录按钮
void CClientDlg::OnBnClickedFileDic()
{
	mymessage send_mss;
	send_mss.type = 0;
	//sendto(c_socket,(char*)&temp,sizeof(message),0,(SOCKADDR*)&addrFrom,len);
	sendto(client->c_socket,(char*)&send_mss,sizeof(mymessage),
    0,(SOCKADDR*)&servAddr,sizeof(SOCKADDR));
	CClientDlg* dlg=(CClientDlg*)AfxGetApp()->GetMainWnd(); //获取主窗口
	dlg->m_log += L"开始交互\r\n";
	dlg->m_log+=L"发送查看共享目录请求\r\n";
	dlg->GetDlgItem(IDC_LOG)->SetWindowTextW(dlg->m_log);
	// TODO: 在此添加控件通知处理程序代码
}
\end{lstlisting}
\end{flushleft}

点击开始上传按钮时,客户端用文件流打开相应文件,客户端便向服务器发送一条type为APPLY\_UPLOAD\_FILE的消息,并将要上传的文件名包含在消息中。并向log中显示对应消息。代码如下:

\begin{flushleft}
\begin{lstlisting}[language={C++},basicstyle=\footnotesize]
//点击开始上传按钮
void CClientDlg::OnBnClickedStartUpload()
{
	CClientDlg* dlg=(CClientDlg*)AfxGetApp()->GetMainWnd(); //获取主窗口
	dlg->m_log += L"开始交互\r\n";
	dlg->m_log+=L"发送上传文件命令\r\n";
	dlg->GetDlgItem(IDC_LOG)->SetWindowTextW(dlg->m_log);
	//用文件流打开所选择的文件
	CString upload_file = dlg->m_upload;
	CStringA file_stra(upload_file.GetBuffer(0));
	upload_file.ReleaseBuffer();
	string s = file_stra.GetBuffer(0);
	const char* file_con = s.c_str();
	if (!(client->u_file = fopen(file_con, "rb")))
	{
		AfxMessageBox(_T("创建本地文件失败!"));
	}
	mymessage send_mss;
	int j = 0;
	for(int i = 0;i<upload_file.GetLength();i++)//获取文件名
	{
		if(upload_file[i] == '\\')
		{
			j = 0;
		}
		send_mss.file_name[j] = upload_file[i];
		j++;
	}
	send_mss.file_name[j] = 0;
	//设置消息类型为申请上传文件
	send_mss.type = APPLY_UPLOAD_FILE;
	send_mss.seq = 0;
	sendto(client->c_socket,(char*)
    &send_mss,sizeof(mymessage),0,(SOCKADDR*)&servAddr,sizeof(SOCKADDR));
	if(dlg->m_upload.GetLength()==0)//如果文件选择栏为空,则弹出对话框
	{
		MessageBox(_T("上传文件为空!"));
	}
	dlg->GetDlgItem(IDC_LOG)->SetWindowTextW(dlg->m_log);
	// TODO: 在此添加控件通知处理程序代码
}
\end{lstlisting}
\end{flushleft}

当用户点击开始下载按钮后,客户端便向服务器端发送一条tpye为APPLY\_DOWNLOAD\_FILE的消息,在这条消息中同样包含了要下载文件的文件名,并且log中显示相应消息,代码如下:

\begin{flushleft}
\begin{lstlisting}[language={C++},basicstyle=\footnotesize]
//点击开始下载按钮
void CClientDlg::OnBnClickedStartDownload()
{
	CClientDlg* dlg=(CClientDlg*)AfxGetApp()->GetMainWnd(); //获取主窗口
	dlg->m_log += L"开始交互\r\n";
	dlg->m_log+=L"发送下载文件命令\r\n";
	mymessage send_mss;
	//设置消息类型为申请下载
	send_mss.type = APPLY_DOWNLOAD_FILE;
	//获取文件名
	for(int i=0;i<m_download.GetLength();i++)
	{
		send_mss.file_name[i] = m_download[i];
	}
	send_mss.file_name[m_download.GetLength()]=0;
	if(dlg->m_download.GetLength()==0)//如果文件选择栏为空,则弹出对话框
	{
		MessageBox(_T("下载文件为空!"));
	}
	char path[MAX_PATH + 1] = { 0 };
	GetModuleFileNameA(NULL, path, MAX_PATH);
	(strrchr(path, '\\'))[0] = 0; // 删除文件名,只获得路径字串
	client->full_ACK = 0;
	strcat(path,"\\");
	client->d_file = fopen(strcat(path,send_mss.file_name),"wb");
	sendto(client->c_socket,(char*)
    &send_mss,sizeof(mymessage),0,(SOCKADDR*)&servAddr,sizeof(SOCKADDR));
	dlg->GetDlgItem(IDC_LOG)->SetWindowTextW(dlg->m_log);
}
\end{lstlisting}
\end{flushleft}


\subsubsection{交互的实现}

当客户端收到type为AGREE\_FILE\_DIC的消息时,客户端便将收到的数据(即共享文件目录)显示在相应窗口中,并在log中添加相应消息。代码如下:

\begin{flushleft}
\begin{lstlisting}[language={C++},basicstyle=\footnotesize]
//服务器同意上传文件
case AGREE_UPLOAD_FILE:
{
	mymessage send_mss;
	//用文件流读文件文件
	int batch_num = fread(send_mss.mss_buffer,1, BUF_SIZE , u_file);
	send_mss.type = START_UPLOAD_FILE;
	int j = 0;
	dlg->m_log += L"上传文件中\r\n";
	CString upload_file = dlg->m_upload;
	send_mss.len = batch_num;
	send_mss.seq = recv.ACK;
	//读出内容长度小于BUF_SIZE,则证明上传完成
	if(batch_num < BUF_SIZE)
	{
		dlg->m_log += L"上传文件完成\r\n";
		send_mss.end_flaaaaaaa[P-=ag = true;
	}
	else
	{
		send_mss.end_flag = false;
	}
	sendto(c_socket,(char*)&send_mss, sizeof(mymessage), 0,
				(SOCKADDR*)&addrFrom, sizeof(SOCKADDR));
	dlg->GetDlgItem(IDC_LOG)->SetWindowTextW(dlg->m_log);
	
	return;
}
\end{lstlisting}
\end{flushleft}


当客户端收到type为AGREE\_UPLOAD\_FILE的消息时,客户端逐包发送带有seq的消息,并将type设置为START\_UPLOAD\_FILE。相应代码如下:

\begin{flushleft}
\begin{lstlisting}[language={C++},basicstyle=\footnotesize]
//服务器同意上传文件
case AGREE_UPLOAD_FILE:
{
	mymessage send_mss;
	//用文件流读文件文件
	int batch_num = fread(send_mss.mss_buffer,1, BUF_SIZE , u_file);
	send_mss.type = START_UPLOAD_FILE;
	int j = 0;
	dlg->m_log += L"上传文件中\r\n";
	CString upload_file = dlg->m_upload;
	send_mss.len = batch_num;
	send_mss.seq = recv.ACK;
	//读出内容长度小于BUF_SIZE,则证明上传完成
	if(batch_num < BUF_SIZE)
	{
		dlg->m_log += L"上传文件完成\r\n";
		send_mss.end_flag = true;
	}
	else
	{
		send_mss.end_flag = false;
	}
	sendto(c_socket,(char*)&send_mss, sizeof(mymessage), 0,
				(SOCKADDR*)&addrFrom, sizeof(SOCKADDR));
	dlg->GetDlgItem(IDC_LOG)->SetWindowTextW(dlg->m_log);
	
	return;
}
\end{lstlisting}
\end{flushleft}

当客户端收到type为END\_UPLOAD\_FILE的消息时,直接跳出。代码如下:
\begin{flushleft}
\begin{lstlisting}[language={C++},basicstyle=\footnotesize]
//上传完成,则跳出
case END_UPLOAD_FILE:
{

	c_posting = false;
	return;
}
\end{lstlisting}
\end{flushleft}


当客户端收到type为AGREE\_DOWNLOAD\_FILE的消息时,客户端便开始下载工作。代码如下:


\begin{flushleft}
\begin{lstlisting}[language={C++},basicstyle=\footnotesize]
//服务器同意下载文件
case AGREE_DOWNLOAD_FILE:
{
	if(recv.seq == -1)
	{
		dlg->m_log += L"开始下载文件\r\n";
	}
	mymessage send_mss;
	send_mss.type = START_DOWNLOAD_FILE;
	//ACK与seq相等,则ACK与len相加后继续发送
	if(full_ACK == recv.seq)
	{
		fwrite(recv.mss_buffer,recv.len,1,d_file);
		dlg->m_log += L"下载文件中\r\n";
		send_mss.type = AGREE_UPLOAD_FILE;
		full_ACK = send_mss.ACK = full_ACK + recv.len;
	}
	else
	{
		send_mss.ACK = full_ACK;
	}
	//结束为为true且不处于开始发送状态,则关闭文件,下载完成
	if(recv.end_flag && recv.seq!=-1)
	{
		fclose(d_file);
		send_mss.type = END_DOWNLOAD_FILE;
		dlg->m_log += L"下载文件完成\r\n";
	}
	else
	{
		send_mss.type = START_DOWNLOAD_FILE;
	}
	sendto(c_socket,(char*)&send_mss,sizeof(mymessage),
    0,(SOCKADDR*)&addrFrom,sizeof(SOCKADDR));
	return;
}
\end{lstlisting}
\end{flushleft}

\subsection{服务器端代码实现}

\subsubsection{总体框架}

在服务器端我们采用同样的框架,在窗口类初始化的时候调用myserver类使其绑定相应端口并处理交互数据。利用与客户端一样的多线程机制使其能够持续监听消息。

\subsubsection{交互的实现}

当服务器端收到type为APPLY\_FILE\_DIC的消息时,之后服务器端查看共享目录的文件列表,将其放在发送报文中,并将其type设置为AGREE\_FILE\_DIC,并将其发送给客户端。代码如下:

\begin{flushleft}
\begin{lstlisting}[language={C++},basicstyle=\footnotesize]
//客户端申请访问共享目录
case APPLY_FILE_DIC:
{
	CServerDlg* dlg=(CServerDlg*)AfxGetApp()->GetMainWnd(); //获取主窗口
	dlg->m_log+=L"收到请求共享目录消息.\r\n";
	send_mss.type = AGREE_FILE_DIC;
	for (int j = 1; j < BUF_SIZE; j++)
	{
		send_mss.mss_buffer[j] = 0;
	}
	char szFilePath[MAX_PATH + 1] = { 0 };
	GetModuleFileNameA(NULL, szFilePath, MAX_PATH);
	(strrchr(szFilePath, '\\'))[0] = 0; // 删除文件名,只获得路径字串
	string p;
	//文件句柄
	long   hFile = 0;
	//文件信息
	struct _finddata_t fileinfo;
	if ((hFile = _findfirst(p.assign(szFilePath)
    .append("\\*").c_str(), &fileinfo)) != -1)
	{
		do
		{
			strncat(send_mss.mss_buffer, 
        strncat(fileinfo.name,"\r\n",5), BUF_SIZE);
		} while (_findnext(hFile, &fileinfo) == 0);
		_findclose(hFile);
	}
	send_mss.mss_buffer[BUF_SIZE-1] = '\n';
	//displayShared(sendBuf);
	sendto(s_socket, (char*)&send_mss, sizeof(mymessage), 
    0,(SOCKADDR*)&clntAddr, sizeof(SOCKADDR));
	end = true;
	dlg->m_log+=L"发送共享文件目录\r\n";
	break;
}
\end{lstlisting}
\end{flushleft}

当服务器端收到type为APPLY\_UPLOAD\_FILE的消息时,客户端便根据发来消息的文件名创建文件,并把回复信息的type设置为AGREE\_UPLOAD\_FILE。代码如下:

\begin{flushleft}
\begin{lstlisting}[language={C++},basicstyle=\footnotesize]
//客户端申请上传文件
case APPLY_UPLOAD_FILE:
{
	CServerDlg* dlg=(CServerDlg*)AfxGetApp()->GetMainWnd(); //获取主窗口
	dlg->m_log+=L"收到请求上传文件消息.\r\n";
	//设置发送信息的类型为AGREE_UPLOAD_FILE
	mymessage send_mss;
	send_mss.type = AGREE_UPLOAD_FILE;
	send_mss.ACK = 0;
	sendto(s_socket, (char*)&send_mss, strLen, 0,
    (SOCKADDR*)&clntAddr, sizeof(SOCKADDR));
	char path[MAX_PATH + 1] = { 0 };
	GetModuleFileNameA(NULL, path, MAX_PATH);
	(strrchr(path, '\\'))[0] = 0; // 删除文件名,只获得路径字串
	full_ACK = 0;
	u_file = fopen(strcat(path,recv_mss.file_name),"wb");
	end = true;
	dlg->m_log+=L"发送同意上传文件消息\r\n";
	break;
}
\end{lstlisting}
\end{flushleft}

当服务器端收到type为START\_UPLOAD\_FILE的消息时,客户端向已创建的文件中写入发送消息包含的文件数据,并根据发过来的seq回复对应的ACK,代码如下:

\begin{flushleft}
\begin{lstlisting}[language={C++},basicstyle=\footnotesize]
//客户端开始上传文件
case START_UPLOAD_FILE:
{
	CServerDlg* dlg=(CServerDlg*)AfxGetApp()->GetMainWnd(); //获取主窗口
	//发送的ACK与初始seq相等,则继续发送
	if(recv_mss.seq == full_ACK)
	{
		fwrite(recv_mss.mss_buffer,recv_mss.len,1,u_file);
		dlg->m_log += L"接收上传文件中\r\n";
		send_mss.type = AGREE_UPLOAD_FILE;
		full_ACK = send_mss.ACK = full_ACK + recv_mss.len;
	}
	//如果不相等,则继续发送上一个ACK
	else
	{
		send_mss.ACK = full_ACK;
	}
	if(recv_mss.end_flag)
	{
		fclose(u_file);
		send_mss.type = END_UPLOAD_FILE;
		dlg->m_log += L"接收上传文件完成\r\n";
	}
	else
	{
		send_mss.type = AGREE_UPLOAD_FILE;
	}
	sendto(s_socket, (char*)&send_mss, 
    sizeof(mymessage), 0,(SOCKADDR*)&clntAddr, sizeof(SOCKADDR));
	end = true;
	break;
}
\end{lstlisting}
\end{flushleft}

当服务器端收到type为APPLY\_DOWNLOAD\_FILE的消息时,打开发送消息所包含的文件名,并将回复信息的type设置为AGREE\_DOWNLOAD\_FILE,代码如下:

\begin{flushleft}
\begin{lstlisting}[language={C++},basicstyle=\footnotesize]
//客户端申请下载文件
case APPLY_DOWNLOAD_FILE:
{
	CServerDlg* dlg=(CServerDlg*)AfxGetApp()->GetMainWnd(); //获取主窗口
	dlg->m_log+=L"收到请求下载文件消息.\r\n";
	mymessage send_mss;
	send_mss.type = AGREE_DOWNLOAD_FILE;
	send_mss.seq = -1;
	sendto(s_socket, (char*)&send_mss, sizeof(mymessage), 
    0,(SOCKADDR*)&clntAddr, sizeof(SOCKADDR));
	dlg->m_log+=L"发送同意下载文件消息.\r\n";
	//dlg->m_log+=CString(recv_mss.mss_buffer);
	char path[MAX_PATH + 1] = { 0 };
	GetModuleFileNameA(NULL, path, MAX_PATH);
	(strrchr(path, '\\'))[0] = 0; // 删除文件名,只获得路径字串
	full_ACK = 0;
	strcat(path,"\\");
	//用文件流打开文件
	d_file = fopen(strcat(path,recv_mss.file_name),"rb");
	end = true;
	break;

}
\end{lstlisting}
\end{flushleft}


当服务器端收到type为START\_DOWNLOAD\_FILE的消息时,服务器读取之前打开的的文件流,并将其放到发送的消息中,并根据ACK发送seq,当读取文件结束时,将结束标志位置1,代码如下:

\begin{flushleft}
\begin{lstlisting}[language={C++},basicstyle=\footnotesize]
//客户端开始下载文件
case START_DOWNLOAD_FILE:
{
	mymessage send_mss;
	int batch_num = fread(send_mss.mss_buffer,1, BUF_SIZE , d_file);
	CServerDlg* dlg=(CServerDlg*)AfxGetApp()->GetMainWnd(); //获取主窗口
	dlg->m_log+=L"开始下载文件.\r\n";
	send_mss.len = batch_num;
	send_mss.seq = recv_mss.ACK;
	//如果文件流读出来的大小小于BUF_SIZE,则证明下载文件完毕,设置结束标志位。
	if(batch_num < BUF_SIZE)
	{
		dlg->m_log += L"下载文件完成\r\n";
		send_mss.end_flag = true;
	}
	else
	{
		send_mss.end_flag = false;
	}
	send_mss.type = AGREE_DOWNLOAD_FILE;
	sendto(s_socket,(char*)&send_mss, sizeof(mymessage), 0,
				(SOCKADDR*)&clntAddr, sizeof(SOCKADDR));
	end = true;
	break;
}
\end{lstlisting}
\end{flushleft}

当服务器端收到type为END\_DOWNLOAD\_FILE的消息时,直接跳出。代码如下:

\begin{flushleft}
\begin{lstlisting}[language={C++},basicstyle=\footnotesize]
//客户端下载结束
case END_DOWNLOAD_FILE:
{
	break;
}
\end{lstlisting}
\end{flushleft}

\subsection{可靠性的保证}

首先我们模仿FTP设置了seq和ACK值只有两者相对应时才能进行传输,从而保证了没有乱序。此外,我们设置了超时重传机制,将超时时间设置为2s,从而超时时,将上一次数据重新发送,代码如下:

\begin{flushleft}
\begin{lstlisting}[language={C++},basicstyle=\footnotesize]
//设置超时重传
int timeout = 2000;      //设置超时时间为2s
setsockopt(s_socket,SOL_SOCKET,SO_RCVTIMEO,(const char*)&timeout,sizeof(int));
recv_mss.type = -1;
\end{lstlisting}
\end{flushleft}

当超时时,上一个要发送的数据则被重新发送。
\clearpage
